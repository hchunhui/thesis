\chapter{线性整数理论命题的证明}
\label{chap:lia}

\section{输入语言}
\begin{figure}[!htbp]
  \centering
  \begin{tabular}[rcl]{rcl}
    & & $C$是整数常元 \\
    & & $X$是整型变元 \\
    $T$ & \sep{} & $C$ \deli{} $X$ \deli{} $+(T, T)$ \deli{} $\cdot(C, T)$ \\
    $Y$ & \sep{} & $T = T$ \deli{} $T \neq T$ \deli{} $T \leq T$ \deli{} $T > T$ \\
    $\Pi$ & \sep{} & $T$ \deli{} $\Pi \land \Pi$ \\
  \end{tabular}
\end{figure}
本部分接收的语言是关于整数环上的线性不等式组,目标是证明不等式组的不可满足性。$\neq$和$>$关系分别是$=$和$\leq$关系的否定形式。

\section{单纯形法}
\subsection{标准型}
单纯形法是线性规划理论中的一种最优化方法。

原始的单纯形法解决的问题的标准型为:
\begin{eqnarray*}
  \max Z = c_1x_1 + c_2x_2 + \cdots + c_nx_n \\
  \begin{cases}
    a_{11}x_1 + a_{12}x_2 + \cdots + a_{1n}x_n = b_1 \\
    a_{21}x_1 + a_{22}x_2 + \cdots + a_{2n}x_n = b_2 \\
    \dots \\
    a_{m1}x_1 + a_{m2}x_2 + \cdots + a_{mn}x_n = b_m \\
    x_j \ge 0, \quad j = 1, 2, \cdots, n \\
    b_i \ge 0, \quad i = 1, 2, \cdots, m \\
  \end{cases}
\end{eqnarray*}
任意的线性规划问题经过适当变换后均可以转换为标准型。

单纯形法在求解时,首先要找到一组初始解,它满足约束条件,但未必最优。接下来,从初始解开始经过一系列迭代,最终达到最优解。如果初始解找不到,则说明问题无解。

单纯形法找初始解的特性为线性不等式组的可满足判定提供了一种天然方法。即若能找到初始解,则不等式组可满足,否则不可满足。

依据上述思路,定义本文中单纯形法解决可满足问题的标准型如下:
\begin{eqnarray*}
  \begin{cases}
    a_{11}x_1 + a_{12}x_2 + \cdots + a_{1n}x_n = s_1 \\
    a_{21}x_1 + a_{22}x_2 + \cdots + a_{2n}x_n = s_2 \\
    \dots \\
    a_{m1}x_1 + a_{m2}x_2 + \cdots + a_{mn}x_n = s_m \\
    x_j \textrm{无限制}, \quad j = 1, 2, \cdots, n \\
    s_i \leq b_i, \quad i = 1, 2, \cdots, m \\
    b_i \in \mathbb{Q}, \quad i = 1, 2, \cdots, m \\
  \end{cases}
\end{eqnarray*}

上述标准型中去掉了优化函数$Z$,因为可满足求解不关心;其次将$x_j$的限制放开,减少变换负担;最后在等式右边加入了$m$个有界的\emph{松弛变量},作用也是减少通常形式到标准型的负担。

至此,所有的不等关系都可以化为适当的标准形式,以$x_1+x_2 \  \mathrm{op} \  1$为例
\begin{itemize}
  \item 若op为$\leq$,标准形式为$x_1+x_2 = s_k, \quad s_k \leq b_k = 1$
  \item 若op为$>$,标准形式为$-1 \cdot x_1 + -1 \cdot x_2 = s_k, \quad s_k \leq b_k = -2$
  \item 若op为$=$,标准形式为$x_1+x_2 = s_k, \quad s_k \leq b_k = 1$和$-1 \cdot x_1 + -1 \cdot x_2 = s_k, \quad s_k \leq b_k = -1$
  \item 若op为$\neq$,分为两个目标,标准形式分别为$x_1+x_2 = s_k, \quad s_k \leq b_k = 0$和$-1 \cdot x_1 + -1 \cdot x_2 = s_k, \quad s_k \leq b_k = -2$
\end{itemize}

\subsection{决策过程}
单纯形法的求解使用单纯形表。单纯形表是一个$m$行$n$列的矩阵外加一个$n$列的向量。矩阵的每一行对应一个\emph{基变量},每一列对应一个\emph{非基变量}。向量每一列代表对应非基变量当前的赋值。

\begin{enumerate}
  \item 初始时,每一行对应松弛变量$s_i$,每一列对应原始变量$x_j$,矩阵是标准型的系数。向量各分量为零。
  \item 接下来,循环求解直到找到可满足实例或不可满足。
    \begin{enumerate}
      \item 首先,检查越界基变量。方法是检查所有代表松弛变量的行$i$,计算$s_i' = \sum_{j=1}^{n} a_{ij} \cdot x_j'$,其中$x_j'$是第$j$列代表的非基变量。若$s_i' > b_i$,$b_i$是松弛变量上界,则说明越界,需要进一步调整。否则,已找到可满足实例,返回``可满足''。
      \item 然后,重新指派非基变量赋值。由于$s_i' = \sum_{j=1}^{n} a_{ij} \cdot x_j'$越上界,因此需要减少某个系数为正的非基变量,或增加某个系数为负的非基变量,最终使得$s_i'$刚好为$b_i$,同时保持所有非基变量也不越界。若找不到,则说明原问题不可满足,返回``不可满足''。
      \item 最后,进行\emph{旋转}操作。旋转操作的目的是交换前两步选中的基变量和非基变量的位置。方法是从$s_i' = \sum_{j=1}^{n} a_{ij} \cdot x_j'$反解出被调整的非基变量$x_k'$,然后代入到其他各行并化简。这步对应矩阵的一系列行变换。
    \end{enumerate}
\end{enumerate}

\subsection{证明项构造}
证明项的构造基于决策过程返回``不可满足''时的单纯形表进行。

证明项构造基于决策过程结束时单纯形表的一个性质:
\begin{theorem}[性质1]
  决策过程返回``不可满足''时,单纯形表的所有非基变量都是松弛变量。
\end{theorem}
\begin{proof}
反设决策过程返回``不可满足''时,存在某一非基变量$x_j'$不是松弛变量。则$x_j'$是原始变量,$x_j'$取值无限制。因此能够通过减小或增大$x_j'$的取值,使得越界基变量刚好不越界。这与返回``不可满足''矛盾。
\end{proof}

\begin{theorem}[性质2]
  决策过程返回``不可满足''时,单纯形表中越界基变量所在行的所有系数为非正。
\end{theorem}
\begin{proof}
反设决策过程返回``不可满足''时,越界基变量行存在某一系数$a_{ij} > 0$。由性质1知,$x_j'$为松弛变量,故无下界。因此可以减小$x_j'$,使得基变量刚好不越界。这与返回``不可满足''矛盾。
\end{proof}

现在可以说明证明的构造方法。\\
设越界变量为$s_i'$,$s_i' = \sum_{j=1}^{n} a_{ij} \cdot x_j'$,$s_i' \leq b_i$。\\
由性质1可保证,对$1 \leq j \leq n$,$x_j'$有上界,即$x_j' \leq b_j'$。\\
由性质2可保证,对$1 \leq j \leq n$,$a_{ij} \leq 0$,即$a_{ij} \cdot x_j' \geq a_{ij} \cdot b_j'$。\\
因此$s_i' = \sum_{j=1}^{n} a_{ij} \cdot x_j' \geq \sum_{j=1}^{n} a_{ij} \cdot b_j' > b_i$,与$s_i' \leq b_i$得到矛盾。

上述过程除了计算之外,需在Coq中定义如下引理:
\begin{verbatim}
volatile : forall x a : Z, x <= a /\ x > a -> False .
\end{verbatim}
应用上述引理即可构造出证明项。

\section{分支限界法}
单纯形法解决的是$\mathbb{Q}$上的线性不等式组的可满足性问题。但是对于一些约束条件,如$4 \leq 3 \cdot x \leq 5$,在$\mathbb{Q}$上可满足,但在我们讨论的$\mathbb{Z}$上是不可满足的。分支限界法用来解决本问题。
\subsection{决策过程}
分支限界法主要通过逐步通过分割可行域,逐步减小搜索范围的办法,将$\mathbb{Q}$上的判定过程转换为$\mathbb{Z}$上的判定过程。
\begin{enumerate}
  \item 首先以原始的不等式组作为输入,调用单纯形法决策过程。
  \item 根据单纯形法返回值:
    \begin{enumerate}
      \item 若为``不可满足'',则$\mathbb{Z}$上也不可满足,返回``不可满足'';
      \item 若为``可满足'',且可满足实例所有分量为整数,则返回``可满足'';
      \item 若为``可满足'',且可满足实例某一分量$x$为分数$\frac{a}{b}$,则分割为两部分:一部分补充不等式$x \leq \lfloor \frac{a}{b} \rfloor$,另一部分补充不等式$x \geq \lceil \frac{a}{b} \rceil$,然后分别递归求解。
    \end{enumerate}
\end{enumerate}

\subsection{证明项构造}
这里证明项的构造是简单的。只要刻画分支的行为即可。

定义如下Coq引理:
\begin{verbatim}
branch : forall x a : Z,
        (x <= a -> False) -> (x > a -> False) -> False.
\end{verbatim}
即可。
