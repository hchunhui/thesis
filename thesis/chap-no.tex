\chapter{理论整合框架的实现}
\label{chap:no}

理论整合框架的作用是,将一个含有多种理论的命题分解成一组命题,两者的可满足性是相同的,且命题组的每一个命题只含某一个理论所能处理的符号。

\section{Nelson-Oppen框架}
Nelson-Oppen框架\cite{Nelson:1979:SCD:357073.357079}是Nelson和Oppen提出的解决理论整合的方案。它的主要步骤如下:
\begin{description}
  \item[纯化] 通过引入新变量,将混合的理论分开。

    如命题$x = 1 \land f(2) = 1 \land \lnot f(+(1,x))=1$经过纯化后,变成$x = u \land f(v) = u \land \lnot f(w)=u$和$u = 1 \land v = 2 \land w = +(1, x)$,前者属于等词与未解释函数理论,后者属于线性整数理论。
  \item[证明] 将纯化公式送往对应理论求解器求解。若某个理论返回``不可满足'',则返回``不可满足''。
  \item[等式传播] 分如下情况:

    若某理论能够推出等式$x = y$,将$x = y$加入其他理论并重启证明。

    若某个理论能够推出$x_1 = y_1 \lor \cdots \lor x_k = y_k$,则分成k种情况,依次将$x_1 = y_1, \dots, x_k = y_k$加入其他理论并证明。若某一种情况返回``可满足'',则返回可满足;否则返回``不可满足''。

    若无法推出新的等式,返回``可满足''。
\end{description}

\section{实现}
Nelson-Oppen框架实现的难点在于等式传播,而其他部分的实现都是直接的。

等式传播的主要问题是,前述的两种理论求解器都只能证实结论,却不能主动推出结论。

如对于命题$x \leq 1 \land 1 \leq x \impl f(x) = f(1)$,经过否定并规范化后得到$x \leq 1 \land 1 \leq x \land \lnot f(x) = f(1)$。我们的期望是$x \leq 1 \land 1 \leq x$送给线性整数求解器得到$x=1$,然后将等式传播给未解释函数求解器,就能得到不可满足的结论。

本文试图采用\emph{推出式}的方法,即改造线性整数求解器,使之能够在某些可满足的情形下附加一组等式,提供闭合可行域中的所有取值,然后一一送给其他求解器求解。这种改造方法在一些特殊的例子下工作得很好,甚至能够解决形如$1 \leq x \land x \leq 3 \land f(1)=1 \land f(2)=2 \land f(3)=3 \impl f(x)=x$这样的``归纳型''命题。

然而,处理命题$x \leq y \land y \leq x \impl f(x) = f(y)$时却遇到困难。原因是条件$x \leq y \land y$确定的区域含有无限个点。若不传播,则命题无法证明;若传播,则会进入死循环。一种办法是让未解释函数求解器也作出提示,如由$\lnot f(x) = f(y)$猜测线性整数证明器应该证明$x=y$。这种方法目前正在继续尝试。

另外还存在\emph{穷举式}的方法,由材料\cite{Harrison:2009:HPL:1540610}给出。它会穷举所有中间结论,并一一尝试证明,因此能够解决推出式的方法。这种方法不需要对理论证明器做修改,模块性较好。这种方法相对于推出式,不能利用具体理论的特性,因此预期性能会下降。

\section{证明项的生成}
Nelson-Oppen框架涉及证明项求解的部分是命题的纯化部分。

仍以
$$x = 1 \land f(2) = 1 \land \lnot f(+(1,x))=1$$
为例说明如何生成证明项。

上述命题在纯化之后,实际证明的是
$$x = u \land f(v) = u \land \lnot f(w)=u \land u = 1 \land v = 2 \land w = +(1, x) \impl false$$

下面构造
\begin{align*}
x = 1 \land f(2) = 1 \land \lnot f(+(1,x))=1 \impl \\
x = u \land f(v) = u \land \lnot f(w)=u \land u = 1 \land v = 2 \land w = +(1, x)
\end{align*}
的证明项,然后应用$\impl$的传递特征,就能得到
$$x = 1 \land f(2) = 1 \land \lnot f(+(1,x))=1 \impl false$$
的证明项。

而证明项可由下述Coq的引理构造:
\begin{verbatim}
eq_ind : forall (A : Type) (x : A) (P : A -> Prop),
               P x -> forall y : A, x = y -> P y .
\end{verbatim}

这就完成了证明项的构造。
