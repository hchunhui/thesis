\chapter{绪论}
\label{chap:intro}

\section{研究背景}
\subsection{程序验证}
计算机技术已经广泛地应用到我们的日常生活中。种类繁多的计算机软件在提供丰富功能的同时,其缺陷也给人们带来困扰。

在一些关键领域,如航天、电信、银行,软件的缺陷可能引发重大的人身和财产损失。因此,提高这些软件的可靠性十分必要。
程序验证(Program Verification)就是一种通过逻辑推理来证明软件具有特定安全性质,从而提高软件可靠性的方法。

1969年,C. A. R. Hoare提出了验证程序正确性的公理系统:Hoare逻辑\cite{Hoare69}。从此Hoare逻辑成为程序验证的重要方法。

\subsection{携带证明的代码}
携带证明的代码(Proof Carrying Code)\cite{Necula97}\cite{Necula98}是由Necula提出的。携带证明的代码携带了机器可检查的证明项(Machine Checkable Proof Term),证明项能够说明代码能满足所需的安全性质。代码消费方仅需用可信任的证明检查器(Proof Checker)去静态地检查证明项,就能够检查程序的安全性。

携带证明的代码的构造给定理证明器提出了新的要求,即定理证明器在宣告命题成立的同时,还要给出相应的机器可检查证明项。

\subsection{分离逻辑}
分离逻辑(Separation Logic)\cite{Reynolds02}是Hoare逻辑的一种扩充。其核心是引入分离合取(Separation Conjuntion)符号,描述两个堆存储区域不相交,从而能够验证一类带有指针的程序。

本文即尝试构造一个用于程序验证的、自动化且出具机器可检查证明项的分离逻辑的定理证明器。由于完整的分离逻辑是不可判定的,因此分离逻辑指的是完整分离逻辑的一个子集。

\section{相关工作}
一阶逻辑 SMT Z3 快速 全自动 不支持分离逻辑

高阶逻辑 Coq 人工交互 表达力强

分离逻辑 Smallfoot ccomp

\section{本文工作及章节安排}
以构造实用的基于分离逻辑的程序验证工具的角度,现有定理证明器的主要问题是:自动化证明能力、表达能力、出具证明能力三者没有作出很好的折中。

Z3是全自动的,但表达能力不足以包括分离逻辑。能够出具证明,但格式为自定义的,不能被广泛信任的Coq证明检查器检查。

Coq被看作一个逻辑框架而被广泛研究和信任,但在此处则显得表达能力过强,导致很多显然的推理无法自动进行。

Smallfoot等虽然能表达分离逻辑,但不能出具证明。由于是用于演示分离逻辑验证的原型系统,在其他方面的证明能力不足。

本文针对这些问题,针对分离逻辑的定理证明,通盘考虑,从头设计与实现一个自动化且出具机器可检查证明项的分离逻辑的定理证明器。

本文的第\ref{chap:struct}章说明证明器的总体设计。第\ref{chap:proof}章给出证明项构造的主要方法。接下来的每一章根据设计框图的划分逐一说明具体的实现方法。最后总结全文。
