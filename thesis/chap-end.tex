\chapter{总结}
\label{chap:end}

\section{工作总结}
本文介绍了一个分离逻辑的定理证明器的设计,其特点是:
\begin{itemize}
  \item 能够支持等词与未解释函数理论、线性整数理论,初步满足程序验证的需要;
  \item 相比Z3等自动证明系统,能够验证基本的分离逻辑定理,并且能够输出Coq兼容的证明项;
  \item 相比Coq等交互式证明系统,具有完全自动化、验证速度快的优势。
\end{itemize}

其中,为了输出Coq兼容的证明项,本文详细地研究了SMT定理证明器每个组件的结构,并针对其特点给出了证明项的构造方法。最终做到了定理证明的每一步都有证明项输出,从而提高了证明的可信度。

根据本文中的设计,本文还初步实现了设计中一阶逻辑的证明部分。通过测试,初步验证了设计的可行性。测试中,证明器能够全自动地证明一阶逻辑命题,并且自动构造Coq兼容的证明项,验证了其可信、自动的特点。

\section{进一步的工作}
进一步的工作可从三方面展开:
\begin{itemize}
  \item 完善决策过程。由于时间所限,目前实现的决策过程都比较简陋,效率还不够高。因此,可以进一步完善现有实现,或者实现效率更高的决策过程。
  \item 改进证明项生成。测试表明,目前的证明项的生成算法用Coq编译时速度不理想,部分测试例生成的证明项过大过复杂。第\ref{chap:test}章的结论部分探讨了改进的方法。
  \item 完成分离逻辑证明器。实现方面,目前分离逻辑部分的证明器还没有做;设计方面,可以进一步考虑加入树、表等归纳谓词,使得证明器实用。
\end{itemize}


